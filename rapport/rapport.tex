\documentclass[a4paper, 11pt, titlepage]{article}
\usepackage{graphicx}
\usepackage{pdfpages}
\usepackage{fancybox}
\usepackage[francais]{babel}
\usepackage[utf8]{inputenc}
% \usepackage[T1]{fontenc}
\usepackage{amsmath,amsfonts,amssymb}
\usepackage{fancyhdr}
\usepackage{stackrel}
\usepackage{babel,indentfirst}
\usepackage{xspace}
\usepackage{url}
\usepackage{titling}
\usepackage{listings}
\usepackage{color}
\usepackage{array}
\usepackage{hyperref}
\usepackage{makecell}
\usepackage{tikz}

%\setlength{\parindent}{0pt}
\setlength{\parskip}{1ex}
\setlength{\textwidth}{17cm}
\setlength{\textheight}{24cm}
\setlength{\oddsidemargin}{-.7cm}
\setlength{\evensidemargin}{-.7cm}
\setlength{\topmargin}{-.5in}




\predate{
\begin{center}
}
\postdate{
\\
\vspace{1.5cm}
\includegraphics[scale=0.7]{imag.png}
\end{center}}


\title {{ {\huge Compte rendu de projet }} }

\author{\Large Equipe 14 \\
\\
    {\sc Aboubacar}~Salim\\
    {\sc Demets}~Jules-Eugène\\
    {\sc Gouttefarde}~Léo\\
    {\sc Rey}~Simon
}

\date{Jeudi 14 Avril 2016}

\lhead{Projet ACVL / Web}
\rhead{Compte rendu}

\begin{document}
\pagestyle{fancy}
\maketitle

\tableofcontents
\newpage

% \begin{center}
% \section* {Introduction }
% \end{center}


% (a) Document d’analyse :
% — acteurs, diagramme de cas d’utilisations et description de ces cas d’utilisations, illustrées par
% des diagrammes de séquence système pertinents.
% — diagramme de classes d’analyse.

\section {Analyse}



% (b) Document de conception :
% — L’architecture générale modèle-vue-contrôleur est imposée, mais indiquez comment vous la
% mettez en œuvre : quels sont les contrôleurs et les vues, comment tout s’articule-t-il.
% — Conception détaillée : diagramme de classes logicielles, diagrammes de séquence, diagrammes
% d’états-transitions si cela est pertinent. Ces diagrammes doivent être cohérents entre eux et
% avec l’implémentation. Il est inutile de fournir des diagrammes illisibles ou qui n’apportent
% aucune information ; il peut être en revanche utile d’ajouter un minimum de texte explicatif
% le cas échéant.

\section {Conception}



\section {Détails techniques}

L'application est normalement protégée contre tout type de faille XSS (via JSTL) et également des attaques par injection SQL.

De plus la vérification des droits d'accès pour chaque type d'action a bien été implémentée, ainsi que la gestion des différentes erreurs dont le message reste intégré au design de l'application.


\section {Manuel d'utilisation}



% (d) Bilan sur les outils de modélisation utilisés, en particulier les problèmes rencontrés, ainsi que
% les solutions trouvées. Il vous est demandé dans cette partie de bien préciser les logiciels, en particulier les modeleurs UML que vous avez utilisés.

\section {Bilan}





\end{document}


